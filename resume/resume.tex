\documentclass[margin,line,pifont,palatino,courier]{res}


\usepackage{longtable}
\usepackage{pifont}
\usepackage[latin1] {inputenc}

\topmargin=-0.45in
\evensidemargin=-.25in
\oddsidemargin=-.25in
\textwidth=5.5in
\textheight=10.0in
\headsep=0.25in

%\topmargin .5in
%\oddsidemargin -.5in
%\evensidemargin -.5in
%\textwidth=6.0in
%\textheight=9.0in
%\itemsep=0in
%\parsep=0in
\usepackage{fancyhdr}
%\topmargin=0in
%\textheight=8.5in
\pagestyle{fancy}
\renewcommand{\headrulewidth}{0pt}
\fancyhf{}
%\cfoot{\thepage}
%\lfoot{\textit{\footnotesize Research Statement}}
\rfoot{{\footnotesize Curriculum Vitae, Vimal Manohar, \thepage}}


\newenvironment{list1}{
  \begin{list}{\ding{113}}{%
      \setlength{\itemsep}{0in}
      \setlength{\parsep}{0in} \setlength{\parskip}{0in}
      \setlength{\topsep}{0in} \setlength{\partopsep}{0in}
      \setlength{\leftmargin}{0.17in}}}{\end{list}}
\newenvironment{list2}{
  \begin{list}{$\bullet$}{%
      \setlength{\itemsep}{0in}
      \setlength{\parsep}{0in} \setlength{\parskip}{0in}
      \setlength{\topsep}{0in} \setlength{\partopsep}{0in}
      \setlength{\leftmargin}{0.2in}}}{\end{list}}

\begin{document}

\name{Vimal Manohar \vspace*{.1in}}

\begin{resume}

\section{\sc Contact Information}

\vspace{.05in}
\begin{tabular}{l l}
The Center for Language and Speech Processing, \\
Hackerman Hall 322,\\
3400 North Charles Street,                        & \verb+vimal.manohar91@gmail.com+\\
Johns Hopkins University,                  & \verb+http://vimalmanohar.github.io+\\
Baltimore, MD 21218, USA               & \\
\end{tabular}

\section{\sc Research Interests}
Speech Processing, Machine Learning, Natural Language Processing

\section{\sc Education}

\textbf{Johns Hopkins University, Baltimore, MD} \\
Major: Electrical \& Computer Engineering \\
Master of Science in Engineering (M.S.E.), 2015\\
Ph.D., Dec. 2018 (Expected) \\
Advisors: Sanjeev Khudanpur and Daniel Povey \\
Thesis topic: Semi-supervised training of acoustic models for speech recognition

\textbf{Indian Institute of Technology Madras, Chennai, India} \\
Major: Electrical Engineering, \quad Minor: Operations Research \\
Bachelor of Technology (B.Tech), 2013 (CGPA: 9.6/10) \\
Advisor: S Umesh, \\

\vspace{-2pt}

\section{\sc Key Publications}
\begin{itemize}
  \item 
  \textbf{Manohar, V.}; Povey, D. et al., `\textit{`Semi-Supervised Training of Acoustic Models using Lattice-Free MMI,"} ICASSP 2018.
  \item
    \textbf{Manohar, V.}; Povey, D. et al., \textit{``JHU Kaldi system for Arabic MGB-3 ASR challenge using diarization, audio-transcript alignment and transfer learning,"} Automatic Speech Recognition and Understanding (ASRU), 2017 IEEE Workshop on. 2017.
  \item 
  Ghahremani, P.; \textbf{Manohar, V.} et al. \textit{``Investigation of Transfer Learning for ASR using LF-MMI Trained Neural Networks,"} Automatic Speech Recognition and Understanding (ASRU), 2017 IEEE Workshop on. 2017.
  \item
    Povey, D.; Peddinti, V.; \textbf{Manohar, V.} et al., \textit{``Purely Sequence-Trained Neural Networks for ASR Based on Lattice-Free MMI,"} Interspeech, pp. 2751-2755. 2016.
  \item
    Peddinti, V.; \textbf{Manohar, V.} et al., 
    \textit{``Far-Field ASR Without Parallel Data,"}
    INTERSPEECH 2016
  \item 
    Liu, C.; Jyothi, P.; \textbf{Manohar, V.} et al.,
    \textit{``Adapting ASR for under-resourced languages using mismatched
    transcriptions,''}
    Acoustics, Speech and Signal Processing (ICASSP), 2016 IEEE Internation Conference on 
  \item 
    Peddinti, V.; Chen, G.; \textbf{Manohar, V.} et al.,
    \textit{``JHU ASpIRE system: Robust LVCSR with TDNNs, iVector adaptation and
    RNN-LMs,''}
    Automatic Speech Recognition and Understanding (ASRU), 2015 IEEE Workshop
    on
  \item 
    \textbf{Manohar, V.}; Povey, D.; Khudanpur, S., 
    \textit{``Semi-supervised Maximum Mutual Information Training of Deep Neural
    Network Acoustic Model,"}
    INTERSPEECH 2015. \textbf{Nominated for best students' paper award}.
  \item
    Trmal, J.; \textbf{Manohar, V.} et al., 
    \textit{``A keyword search system using open source software," } 
    Spoken Language Technology Workshop (SLT), 2014 IEEE, pp.530,535
  \item
    \textbf{Manohar, V.}; Srinivas, C.B.; Umesh, S., 
    \textit{``Acoustic modeling
    using transform-based phone-cluster adaptive training,"} 
    Automatic Speech Recognition and Understanding (ASRU), 2013 IEEE Workshop on
    , pp.49,54
\end{itemize}
\section{\sc Research and Industrial Experience}

%\begin{tabular}{p{0.2\columnwidth} p{0.8\columnwidth}}
%June -- Aug '16 & \begin{minipage}{0.8\columnwidth}
%    \textbf{Research Intern at Microsoft Research in Speech and Dialog Group} \\
%    Mentor: Mike Seltzer
%\end{minipage}
%\end{tabular}

\textbf{Intern at Microsoft Research in Speech and Dialog Group} \\
Mentor: Mike Seltzer \hfill 
June -- August '16 \vspace{2pt}
%Worked on robust speech recognition and lightly-supervised ASR.

\textbf{Jelinek Summer Workshop on Speech and Language Technology (JSALT) 2015 } \\
University of Washington Seattle, Seattle, WAS, USA \hfill 
July -- August '15 \vspace{2pt} \\
Member of the research group working on ``Probabilitic Transcription of Languages with no native-language transcribers''. We showed the utility of mismatched transcriptions from non-native crowdworkers for ASR.
\vspace{-5pt}

\textbf{Research Assistant at the Center for Language and Speech Processing} \\
Johns Hopkins University, Baltimore, MD, USA \hfill Aug '13 -- Present \vspace{2pt} \\
\textit{MGB-3 Challenge 2017} \\
Worked on speaker diarization, lightly-supervised ASR and transfer learning across domains and dialects (ASRU 2017)

\textit{NIST OpenSAT 2017} \\
Worked on neural network-based speech activity detection using LSTM and statistics pooling for long temporal context

\textit{IARPA Babel} \\
Low-resource ASR, speech segmentation, semi-supervised training for ASR 
%Developed acoustic models for languages in low-resource setting, HMM-GMM based automatic
%speech segmentation for ASR, semi-supervised training
%approaches for hybrid HMM-DNNs and bottleneck feature NNs \\ 
%(published in SLT, 2014). \\

\textit{DARPA BOLT}\\
Multilingual DNN for transfer learning across dialects
% multilingual-architecture DNN systems for transfer learning from standard Arabic to Egyptian Arabic\\

\textbf{Intern at Analog Devices Inc.} \\
Cambridge, MA, USA\hfill May -- Aug '14 \vspace{2pt} \\
Worked on time-frequency masks with multichannel audio for robust speech recognition
\vspace{-5pt}

\textbf{Bachelor's Thesis Project} \\
Indian Institute of Technology Madras, Chennai, India\hfill Sept '12 -- May '13 \vspace{2pt}  \\
Proposed phone cluster-adaptive training model for low-resource ASR. (ASRU, 2013)

%\textbf{Time-scaling and Pitch-scaling of synthesized speech} \\
%Indian Institute of Technology Madras, Chennai, India \hfill March -- May '12 \vspace{2pt} \\
%Investigated algorithms for robust VAD, robust pitch estimation, pitch-mark extraction, pitch synchronous overlap-add method of speech synthesis to change the duration and pitch of speech signals
%\vspace{-5pt}

\textbf{Research Intern at The Institute of Automation}  \\
University of Bremen, Bremen, Germany\hfill May -- July '12 \vspace{2pt}\\
Worked on modeling 3D objects from stereo images. %Implemented a method for estimation of size, position and orientation of
%isolated 3D objects using a single pair of stereo images \\
%\vspace{-5pt}

\textbf{Texas Instruments Analog Design Contest 2011} \\
Indian Institute of Technology Madras, Chennai, India \hfill Sept '11 -- Feb '12 \vspace{2pt} \\
Designed and constructed a pulse oximeter on an embedded system for real-time estimation of respiratory rate. Among the top 25 entries to the TI India Analog Design Contest 2011.\\
\vspace{-5pt}

\section{\sc Teaching Experience}
\begin{tabular}{@{}p{0.9in} p{4in}}
Fall 2015 & Teaching Assistant, Random Signal Analysis \\
\end{tabular}


%\section{\sc Coursework}
%
%\begin{tabular}{@{}p{2.3in}p{3in}}
%\begin{list1}
%\item Representation learning
%\item Random Signal Analysis
%\item Natural language processing
%\item Speech and audio processing by humans and machines
%\item Non-linear optimization
%\item Information Extraction from text and speech
%\item Compressed Sensing and Sparse Recovery
%
%\end{list1}
%&
%\begin{list1}
%\item Information Theory
%\item Speech Technology
%\item Matrix Analysis
%\item Graph Theory
%\item Advanced Operations Research
%\item Computer Vision
%\item Advanced DSP
%\end{list1}
%
%\end{tabular}

%\section{\sc Extra-\\ Curricular \\ Activities}
%
%    \textbf{Robotics} 
%\begin{itemize} \itemsep -2pt
%        \item Member of the team representing IIT Madras at the National Robotic Contest, Abu Robocon 2011. The team finished among the Top 5 in the country.
%        \item Winner of Autonomous Robotics and Image Processing Robotics competitions held at Techfest 2012, the technical festival of NIT Trichy
%    \end{itemize}
%
%    \textbf{Electronics} 
%\begin{itemize} \itemsep -2pt
%        \item Designed an accelerometer-gyroscore-magnetometer-based \emph{3D mouse} for controlling 3D CAD objects. The design won \textit{the GE Industrial Defined Problem} at Shaastra 2012, the technical festival of IIT Madras. 
%    \end{itemize}
%
%    \textbf{Community Service} 
%\begin{itemize} \itemsep -2pt
%    \item Volunteer for Association for India Development (AID), a charity
%      organization supporting sustainable development projects in India.
%    \item Volunteered for National Social Service (NSS) at IIT Madras (2009-10). Devised scientific experiments and models and created scientific website content for teaching secondary school students.
%  \end{itemize}

\section{\sc Distinctions}
\begin{itemize} \itemsep -2pt
    \item Alexa Graduate Fellowship 2018
    \item ECE Graduate Fellowship 2013, Johns Hopkins University
    \item Hamburger Fellowship 2013, Johns Hopkins University
    \item WISE Scholarship 2012, DAAD, Germany
    \item All India Rank \textbf{191} in {IIT-Joint Entrance Examination (IIT-JEE)} 2009% (among over 400,000 students) 
    \item Kishore Vaignayik Protsahan Yojana (KVPY) Fellowship 2008, Govt. of India % by Dept. of Science and Technology, Govt. of India
    \item National Talent Search (NTS) Scholarship 2007, Govt. of India %by National Council of Education, Research and Training, Govt. of India
    \item Member of IIT Madras team at the National Robotics Contest, Abu Robocon 2011. Placed among the Top 5 in India
    %\item Winner of Autonomous robotics and Image processing robotics at Techfest 2012, NIT Trichy, India
    %\item Winner of GE Industrial Defined Problem at Shaastra 2012, IIT Madras, India for design of accelerometer-gyroscope-magnetometer-based 3D-mouse
  \end{itemize}

\section{\sc Skills}

\begin{tabular}{@{}p{0.8in}p{6in}}

Languages:& C/C++, Python, Bash, MATLAB\\
Toolkits: & KALDI, HTK, CNTK \\

\end{tabular}

\section{\sc References}

Will be provided on request.

\end{resume}
\end{document}
